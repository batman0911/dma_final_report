\section{Áp dụng nhiễu vào hàm mục tiêu}
Như thuật toán chèn đã được trình bày khá nông và còn thiếu sót, tác giả tin rằng nên thêm yếu tố ngẫu nhiên vào heuristic chèn sao cho không phải lúc nào cũng thực hiện chuyển động tốt nhất trong phần tối ưu cục bộ. Điều này đạt được bằng cách thêm nhiễu vào hàm mục tiêu. Mỗi lần tính toàn chi phí $C$ của 1 vòng lặp của 1 yêu cầu trong tuyến đường, ta cũng tính 1 số ngẫu nhiên nhiễu (noise) có giá trị trong khoảng $\left[ -maxN, maxN \right]$ và tính chi phí thêm $C'=max\{ 0, C+noise \}$. Tại mỗi vòng lặp ta quyết định nếu sử dụng \textit{C} hay \textit{C'} để làm rõ tính hiệu quả của vòng lặp. Quyết định này được thực hiện bởi cơ chế thích nghi được mô tả ở phần trước bằng cách theo dõi cách mà nhiều thường tác động lên việc chèn và các lần chèn hiệu quả.

Để tạo ra số lượng nhiễu liên quan đến các thuộc tính của bài toán, ta tính toán $maxN = \eta\cdot max_{i,j \in V} \{d_{ij}\}$, trong đó $\eta$ là tham số điều khiển lượng nhiễu. Ta chọn $maxN$ độc lập với khoảng cách $d_{ij}$ là 1 phần quan trọng của mục tiêu trong tất cả các bài toán được xem xét trong paper này.

Có vẻ như không cần thiết để thêm nhiễu vào insertion heuristic, vì kỹ thuật này được sử dụng trong mô phỏng luyện kim vì nó đã có sẵn sự ngẫu nhiên. Tuy nhiên, tác giả tin rằng sự ứng dụng của nhiễu quan trọng vì vùng lân cận được tìm kiếm bởi trung bình của các heuristic chèn và không được lấy mẫu ngẫu nhiên. Nếu không có các ứng dụng của nhiễu, ta không nhận được đầy đủ lợi ích của siêu giả lập luyện kim. Phỏng đoán này được hỗ trợ bởi các thí nghiệm tính toán trong Bảng 4.
