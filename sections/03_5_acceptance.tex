\section{Tiêu chí chấp nhận và dừng}
Như đã trình bày trong Phần 2, 1 tiêu chí đơn giản có thể dùng là chỉ chấp nhận các phương án tốt hơn hiện tại. Điều này cho chúng ta 1 phương pháp descent heuristic giống như phương pháp đã trình bày bởi Shaw(1997). Tuy nhiên, mỗi heuristic có 1 khuynh hướng bị mắc kẹt trong 1 tối ưu cục bộ, nên có vẻ hợp lý khi chấp nhận các phương án tồi tệ hơn phương án hiện tại. Để làm điều này, tác giả sử dụng tiêu chí chấp nhận từ mô phỏng luyện kim (simulated annealing). Nghĩa là ta chấp nhận 1 phương án \textit{s'} cho phương án hiện tại \textit{s} với xác suất $e^{-(f(s')-f(s))/T}$ với \textit{T} > 0 là nhiệt độ (\textit{temperature}).

Nhiệt độ được bắt đầu từ $T_{start}$ và giảm ở mỗi vòng lặp theo biểu thức $T=T \cdot c$, trong đó $0<c<1$ là tỷ lệ làm lạnh \textit{cooling rate}. 1 lựa chọn tốt của $T_{start}$ là độc lập với bài toán thực hiện, do đó thay vì xác định rõ $T_{start}$ như là 1 tham số, ta tính toán nó bằng cách theo dõi phương án ban đầu (\textit{initial solution}). Đầu tiên chi phí $z'$ của phương án được tính toán dùng 1 hàm mục tiêu đã được sửa đổi. Trong hàm mục tiêu này, $\gamma$ (chi phí của việc có yêu cầu trong hàng chờ) được đặt là 0. Nhiệt độ bắt đầu lúc này được đặt sao cho phương án kém hơn $\omega\%$ phương án hiện tại được chấp nhận với xác suất là 0.5. Lý do cho việc đặt $\gamma$ là 0 vì tham số này thường lớn và có thể gây ra việc nhiệt độ ban đầu quá lớn nếu phương án ban đầu có vài yêu cầu trong hàng chờ. $\omega$ hiện tại là 1 tham số cần được khởi tạo. Ta gọi tham số này là tham số điều khiển nhiệt độ bắt đầu (\textit{start temperature control parameter}). Thuât toán dừng lại khi 1 số vòng lặp nào đó của LNS được thực thi xong.
