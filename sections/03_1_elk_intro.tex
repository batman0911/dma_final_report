\section{Giới thiệu ELK-stack}
ELK Stack là từ viết tắt được sử dụng để mô tả bộ dịch vụ bao gồm 3 dự án phổ biến: Elasticsearch, Logstash và Kibana. Thường được gọi là Elasticsearch, ELK Stack mang tới khả năng tổng hợp log từ tất cả các hệ thống và ứng dụng của bạn, phân tích những log này, hiển thị dữ liệu để giám sát ứng dụng và cơ sở hạ tầng, khắc phục sự cố nhanh hơn, phân tích bảo mật, v.v. Không quan trọng webserver sử dụng NGINX, Apache hay MSServer, không quan trọng log đến từ ứng dụng, database, cache server... chúng đều có thể được xử lý bởi ELK!

\begin{figure}[H] % places figure environment here   
    \centering % Centers Graphic
    \includegraphics[width=0.8\textwidth]{figures/elk_02.png} 
    \caption{ELK stack} % Creates caption underneath graph
    \label{fig:elk_02}
\end{figure}

\begin{itemize}
    \item \textit{E = Elasticsearch}: Elasticsearch là công cụ tìm kiếm và phân tích phân tán được xây dựng trên Apache Lucene. Khả năng hỗ trợ đa dạng ngôn ngữ, hiệu suất cao và JSON doccument phi cấu trúc khiến Elasticsearch trở thành một lựa chọn lý tưởng cho nhiều trường hợp sử dụng tìm kiếm và phân tích log khác nhau.
    \item \textit{L = Logstash}: Logstash là một công cụ thu nạp dữ liệu nguồn mở cho phép thu thập dữ liệu từ các nguồn khác nhau, chuyển đổi dữ liệu và gửi dữ liệu tới điểm đích. Với các bộ lọc được tạo sẵn và hỗ trợ hơn 200 phần bổ trợ, Logstash cho phép người dùng dễ dàng thu nạp bất kỳ dữ liệu đến từ nguồn hay thuộc loại dữ liệu nào.
    \item \textit{K = Kibana}: Kibana là một công cụ hiển thị trực quan và khám phá dữ liệu dành cho hoạt động đánh giá log và sự kiện. Kibana cung cấp các biểu đồ tương tác dễ sử dụng, các bộ lọc được tạo sẵn cũng như hỗ trợ không gian địa lý. 
\end{itemize}

\begin{figure}[H] % places figure environment here   
    \centering % Centers Graphic
    \includegraphics[width=0.6\textwidth]{figures/elk_01.png} 
    \caption{ELK stack với log ứng dụng} % Creates caption underneath graph
    \label{fig:elk_01}
\end{figure}