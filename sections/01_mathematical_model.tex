\chapter{Mô hình toán học}
Phần này trình bày một mô hình toán học của bài toán; nó dựa trên mô hình được đề xuất bởi Desaulniers et al. (2002). Mô hình toán học phục vụ như một mô tả chính thức của vấn đề. Khi chúng ta giải bài toán heuristically, chúng ta không cố gắng viết mô hình ở dạng integer-linear.

Một ví dụ về bài toán nhận và giao hàng chứa n yêu cầu và m phương tiện. Bài toán được xác định trên đồ thị, \textit{P = \{1+,...,n\}} là tập các nút đón, \textit{D=\{n+1,...,2n\}} là tập hợp các nút phân phối. Yêu cầu i được đại diện bởi các nút i và i + n. K là tập hợp tất cả các phương tiện \textit{|K|=m}. Một phương tiện có thể không phục vụ được tất cả các yêu cầu; chẳng hạn, một yêu cầu có thể yêu cầu xe phải có ngăn cấp đông.
$K_i$ là tập hợp các phương tiện có khả năng phục vụ yêu cầu i, và $P_k \subseteq P$ và $D_k \subseteq D$ lần lượt là tập hợp các hoạt động nhận hàng và giao hàng có thể được phục vụ bởi phương tiện k; do đó với tất cả phương tiện i và k: $k \in K_i \Leftrightarrow i \in P_k \land i \in D_k $. Yêu cầu mà $K_i \ne K$ được gọi là \textit{special requests}. Xác định $N = P \cup D$ và $N_k = P_k \cup D_k$. Cho $\tau_k = 2n +k, k \in K$, và cho $\acute{\tau_k} = 2n + m +k, k \in K$ lần lượt là các nút đại diện cho bến đầu và bến cuối của xe k. 
Đồ thị $G = (V,A)$ bao gồm các nút $V= N \cup \{ \tau_1,...,\tau_m\} \cup \{\acute{\tau_1},...,\acute{\tau_m}\}$ và các cung $A = V \times V$. Mỗi phương tiên, chúng ta có một đồ thị con $G_k = (V_k, A_k)$, với $V_k = N_k \cup \{\tau_k\} \cup \{\acute{\tau_k}\}$ và $A_k =V_k \times V_k$. Với mỗi cạnh $(i, j) \in A$ chúng ta gán khoảng cách $d_{ij} \geq 0$ và thời gian di chuyển $t_{ij} \geq 0$. Giả sử rằng khoảng cách và thời gian là không âm; $d_{ij} \geq 0, t_{ij} \geq 0$ và số lần thỏa mãn bất đẳng thức tam giác; $dt_{ij} \leq t_{il} + t_{lj}$ với mọi $i, j, l \in V$.
Để mô hình hóa, chúng ta cũng giả sử rằng $t_{i,n+i} + s_i > 0$; điều này làm cho việc loại bỏ các chuyến phụ và ràng buộc nhận hàng trước khi giao hàng trở nên dễ dàng để mô hình hóa.

Mỗi nút $i \in V$ có thời gian phục vụ $s_i$ và thời gian ràng buộc $[a_i, b_i]$ . Thời gian phục vụ thể hiện thời gian cần thiết để bốc dỡ hàng và thời gian ràng buộc cho biết khi nào chuyến thăm địa điểm cụ thể phải bắt đầu; một lần truy cập vào nút i chỉ có thể diễn ra giữa thời gian $a_i$ và $b_i$. Một phương tiện được phép đến một địa điểm trước khi bắt đầu thời gian ràng buộc, nhưng nó phải đợi cho đến khi bắt đầu thời gian ràng buộc trước khi chuyến thăm có thể được thực hiện. Với mỗi nút $i \in N, l_i$ là số lượng hàng hoá phải trất lên phương tiện tại nút cụ thể, $l_i \geq 0$ với $i \in P$ và $l_i = -l_{i-n}$ với $i \in D$. Công suất của mỗi phương tiện $k \in K$ ký hiệu là $C_k$.

Bốn loại biến quyết định được sử dụng trong mô hình toán học. $x_{ijk}, i, j \in V, k \in K$ là một biến nhị phân là 1 nếu cạnh giữa nút i và nút j được sử dụng bởi xe k và 0 nếu ngược lại. 
$S_{ik}, i \in V, k \in K$ là một số nguyên không âm cho biết thời điểm xe k bắt đầu dịch vụ tại địa điểm i, $L_{ik}, i \in V, k \in K$ là số nguyên không âm là cận trên của lượng hàng hóa trên xe k sau khi phục vụ tại nút i. $S_{ik}$ và $L_{ik}$ chỉ được xác định rõ khi xe k thực sự ghé qua nút i.
Cuối cùng $z_i i \in P$ là biến nhị phân cho biết yêu cầu i có được đặt trong \textit{request bank} hay không. Biến là một nếu yêu cầu được đặt trong \textit{request bank} và bằng không nếu không.

\begin{equation} \label{eq1}
    min \alpha \sum_{k \in K} \sum_{(i,j) \in A} d_{ij}x_{ijk} + \beta \sum_{k \in K}(S_{\acute{\tau_k},k}- \alpha_{\tau k}) + \gamma\sum_{i \in P}z_i 
\end{equation}
Trong đó:
\begin{flalign}
    \label{ct:2} &\sum_{k \in K_i} \sum_{j \in N_k} x_{ijk} + z_i = 1 \quad\forall i \in P& \\ 
    \label{ct:3}&\sum_{j \in V_k} x_{ijk} - \sum_{j \in V_k} x_{j,n+i,k}=0 \quad \forall k \in K, \forall i \in P_k&\\
    \label{ct:4}&\sum_{j \in P_k \cup \{\acute{\tau_k}\}}x_{\tau_k,j,k}=1 \quad \forall k \in K&\\
    \label{ct:5}&\sum_{i \in D_k \cup \tau_k} x_{i,\acute{\tau_k},k}=1 \quad \forall k \in K&\\
    \label{ct:6}&\sum_{i \in V_k}x_{ijk}-\sum_{i \in V_k}x_{jik}=0\quad \forall k \in K, \forall j \in N_k&\\
    \label{ct:7}&x_{ijk}=1 \Rightarrow S_{ik}+s_i+t_{ij}<S_{jk}\quad \forall k \in K, \forall (i,j) \in A_k&\\
    \label{ct:8}&a_i \geq S_{ik} \geq b_i \quad \forall k \in K, \forall i \in V_k&\\
    \label{ct:9}&S_{ik}\geq S_{n+i} \quad \forall k \in K, \forall i \in P_k&\\
    \label{ct:10}&x_{ijk}=1 \Rightarrow L_{ik}+l_j \leq L_{jk} \quad \forall k \in K, \forall (i,j) \in A_k&\\
    \label{ct:11}&L_{ik}\leq C_k \quad \forall k \in K, \forall i \in V_k&\\
    \label{ct:12}&L_{\tau_k k}=L_{\acute{\tau_k}k} \quad \forall k \in K&\\
    \label{ct:13}&x_{ijk} \in \{0,1\} \quad \forall k \in K, \forall (i,j) \in A_k&\\
    \label{ct:14}&z_i \in \{0,1\} \quad \forall i \in P &\\
    \label{ct:15}&S_{ik} \geq 0 \quad \forall k \in K, \forall i \in V_k&\\
    \label{ct:16}&L_{ik} \geq 0 \quad \forall k \in K, \forall i \in V_k
\end{flalign}

Hàm mục tiêu giảm thiểu tổng trọng số của quãng đường đã đi, tổng thời gian mà mỗi phương tiện sử dụng và số lượng yêu cầu không được lên lịch.

Ràng buộc (\ref{ct:2}) đảm bảo rằng mỗi địa điểm nhận được truy cập hoặc yêu cầu tương ứng được đặt trong ngân hàng yêu cầu. Ràng buộc (\ref{ct:3}) đảm bảo rằng địa điểm giao hàng được truy cập nếu địa điểm nhận hàng được truy cập và chuyến thăm đó được thực hiện bởi cùng một phương tiện. Các ràng buộc (\ref{ct:4}) và (\ref{ct:5}) đảm bảo rằng một phương tiện rời khỏi mọi bến đầu và một phương tiện đi vào mọi bến cuối. Cùng với ràng buộc (\ref{ct:6}), điều này đảm bảo rằng các đường dẫn liên tiếp giữa $\tau_k$ và $\acute{\tau_k}$ được hình thành cho mỗi $k \in K$.

Các ràng buộc (\ref{ct:7}) và (\ref{ct:8}) đảm bảo rằng $S_ik$ được đặt chính xác dọc theo các con đường và các thời gian ràng buộc được tuân theo. Những ràng buộc này cũng làm cho các \textit{subtour} không thể thực hiện được.
Ràng buộc (\ref{ct:9}) đảm bảo rằng mỗi lần lấy hàng xảy ra trước lần giao hàng tương ứng. Các ràng buộc (\ref{ct:10}), (\ref{ct:11}) và (\ref{ct:12}) đảm bảo rằng biến tải trọng được đặt chính xác dọc theo các con đường và các ràng buộc về sức chứa của các phương tiện được tuân thủ
