\section{Các phương pháp thêm yêu cầu}
Các kỹ thuật thêm cho các bài toán định tuyến thường được chia thành 2 dạng: tuần tự và song song. Điểm khác biệt giữa 2 dạng này là phương pháp tuần tự xây dựng 1 tuyến đường chỉ trong 1 thời điểm còn phương pháp song song xây dựng các tuyến đường cùng lúc. 2 phương pháp này được trình bày chi tiết trong Potvin and Rousseau (1993). Kỹ thuật được trình bày trong bài này là kỹ thuật song song. Người đọc nên quan sát rằng kỹ thuật thêm được trình bày ở đây được sử dụng trong bối cảnh mà chúng được cung cấp 1 số tuyến đường 1 phần và 1 số yêu cầu để chèn hơn là việc xây dựng thuật toán từ đầu.

\subsection{Phương pháp heuristic tham lam cơ bản}
Heuristic tham lam cơ bản là 1 kỹ thuật xây dựng đơn giản. Nó thực hiện tối đa \textit{n} lần lặp vì nó chèn thêm 1 yêu cầu trong mỗi lần lặp. Với $\Delta f_{i, k}$ biểu diễn cho sự thay đổi trong giá trị phát sinh của mục tiêu bằng cách chèn thêm yêu cầu \textit{i} vào tuyến đường \textit{k} tại vị trí mà giá trị mục tiêu là nhỏ nhất. Nếu không chèn yêu cầu \textit{i} và tuyến đường \textit{k} thì ta đặt $\Delta f_{i, k} = \infty$ và $c_i = min_{k \in K}\{\Delta f_{i, k}\}$ . Nói cách khác, $c_i$ là chi phí khi chèn thêm yêu cầu \textit{i} vào vị trí tốt nhất trong giải pháp, gọi là vị trí chi phí nhỏ nhất. Cuối cùng, ta chọn yêu cầu \textit{i} với: $min_{i \in U} c_i$ và chèn nó vào vị trí chi phí nhỏ nhất. \textit{U} là tập các yêu cầu chưa được lên kế hoạch. Quá trình này tiếp tục cho đến khi tất cả các yêu cầu được thêm hoặc không còn yêu cầu nào nữa.

Nhận thấy trong mỗi vòng lặp, thuật toán chỉ thay đổi 1 tuyến đường (tuyến mà yêu cầu mới được thêm vào), và thuật toán không cần phải tính toán lại chi phí chèn thêm trong tất cả các tuyến đường khác. Điểm này giúp tăng hiệu năng cho thuật toán.

Dễ dàng nhận thấy 1 vấn đề với cách tiếp cận này là nó thường trì hoãn việc đặt các yêu cầu có chi phí cao cho các lần lặp cuối cùng, nơi chúng ta không có nhiều cơ hội cho việc chèn thêm yêu cầu vì nhiều tuyến đường đều đã kín. Phương pháp heuristic được trình bày trong phần sau sẽ cố gắng giải quyết hạn chế này.

\subsection{Phương pháp regret heuristic}
Phương pháp này cố gắng cải thiện nhược điểm của kỹ thuật tham lam bằng cách kiểm tra lại kết quả sau khi chọn chèn thêm yêu cầu. Đặt $x_{ik} \in {1, ..., m}$ là biến biểu diễn tuyến đường cho yêu cầu \textit{i} có chi phí chèn thêm vào thấp thứ \textit{k}, điều này có nghĩa là $\Delta f_{i, x_{ik}} \leqslant \Delta f_{i, x_{ik'}}$. Sử dụng ký hiệu này, ta có thể biểu diễn $c_i$ từ Phần 3.2.1 như sau: $c_i = \Delta f_{i, x_{i1}}$. Trong phương pháp này, chúng ta có thể định nghĩa 1 giá trị \textit{regret} $c_i^* = \Delta f_{i, x_{i2}} - \Delta f_{i, x_{i1}}$ . Nói cách khác, giá trị regret là khoảng cách giữa chi phí của việc chèn thêm yêu cầu vào tuyến đường tốt nhất so với tuyến đường tốt thứ 2 tốt thứ 2. Trong mỗi vòng lặp, thuật toán chọn ra yêu cầu \textit{i} thỏa mãn điều kiện: $max_{i \in U} \{C_i^*\}$. Với điều kiện này, yêu cầu được đảm bảo thêm vào vị trí có chi phí nhỏ nhất. Điều này khiến cho các ràng buộc bị phá vỡ. Nói cách khác, thuật toán sẽ thực hiện chèn yêu cầu vào giải pháp nếu không sẽ hối tiếc khi không thực hiện điều này.

Phương pháp này có thể mở rộng 1 cách tự nhiên để định nghĩa 1 lớp các phương pháp regret heuristic: phương pháp k-regret heuristic là 1 phương pháp mà mỗi lần thêm yêu cầu vào giải pháp cần phải thỏa mãn điều kiện:
\begin{equation}
    \max\limits_{i \in U} \{ \sum_{j=1}^k (\Delta f_{i, x_{ij}} - \Delta f_{i, x_{i1}}) \}
\end{equation}
Nếu 1 vài yêu cầu không thể được chèn thêm vào ít nhất $m-k+1$ tuyến đường thì yêu cầu đó sẽ được chèn vào số lượng tuyến đường ít nhất (có thể là 1). Các ràng buộc bị phá vỡ bởi việc lựa chọn yêu cầu với chi phí chèn tốt nhất. Yêu cầu được chèn vào vị trí ít chi phí nhất. Thuật toán heuristic hối tiếc được trình bày trong Phần 3.2.1 là 1-regret heuristic bởi vì cách phá vỡ quy tắc. Nói cách khác, với k>2 thì thuật toán sẽ tính toán chi phí của việc thêm vào 1 yêu cầu qua k tuyến đường tốt nhất và chèn yêu cầu mà khoảng cách chi phí giữa việc thêm nó vào tuyến đường tốt nhất với tuyến đường tốt thứ k-1 là lớn nhất. So sánh với 2-regret heuristic, thuật toán với giá trị lớn của \textit{k} khám phá ra sớm hơn khả năng bị giới hạn khi thêm 1 yêu cầu.

Heuristic hối tiếc đã được sử dụng bởi Potvin và Rousseau (1993) cho VRPTW. Kỹ thuật trong bài báo của họ có thể được phân loại vào k-regert heuristic với k=m, vì tất cả các tuyến đường được xem xét trong 1 biểu thức tương tự (3.2). Nhóm tác giả không sử dụng sự thay đổi trong giá trị mục tiêu để đánh giá chi phí của 1 lần chèn mà sử dụng 1 hàm chi phí đặc biệt. Các reget heuristic cũng có thể được sử dụng cho các vấn đề tối ưu hóa tổ hợp bên ngoài miền định tuyến phương tiện, một ví dụ về ứng dụng cho vấn đề chuyển nhượng tổng quát được trình bày bởi Martello và Toth (1981). Như trong phần trước, có 1 sự thật là chỉ 1 tuyến đường được thay đổi trong mỗi vòng lặp để tăng tốc kỹ thuật regret heuristic.