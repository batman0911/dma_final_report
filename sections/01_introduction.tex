\chapter{Giới thiệu}
Trước hết, báo cáo này không phải một hướng dẫn cụ thể về cài đặt công cụ hay về một chuẩn mực nào đó trong thiết kế hay phát triển phần mềm. Nội dung báo cáo dựa trên kinh nghiệm thực tế của tác giả trong quá trình làm sản phẩm (phần mềm), vậy nên chúng tôi sẽ trình bày một cách \textit{vui vẻ} và \textit{gần gũi} thay vì sử dụng ngôn ngữ \textit{hàn lâm}.

Có nhiều tiêu chuẩn để đánh giá một phần mềm là tốt hay không, ví dụ như hiệu năng cao, ít lỗi, sử dụng tài nguyên hiệu quả... Mỗi người hoặc mỗi sản phẩm có đặc thù riêng để đặt ra tiêu chuẩn riêng. Ngoài ra cũng không tồn tại một sản phẩm nào mà vừa phát triển nhanh vừa chạy ổn định (ít hoặc không có lỗi) lại có hiệu năng cao và sử dụng ít tài nguyên, chúng ta luôn phải đánh đổi. Nhưng điều đó không có nghĩa là chúng ta phải bỏ đi tiêu chuẩn nào đó mà không cố gắng làm cho phần mềm ngày một tốt hơn. Trước khi làm được điều đó, chí ít chúng ta cần phải nắm được phần mềm đang tệ hoặc tốt ở điểm nào. Giám sát hệ thống sinh ra từ đó. 

Một hệ thống giám sát, cảnh báo tốt có khả năng chỉ ra những thông tin có giá trị một cách trực quan nhất để người quản trị có thể dựa vào đó để đưa ra định hướng phát triển hoặc sửa chữa những lỗi xảy ra trong phần mềm. Hơn nữa, khi đặt các ngưỡng cảnh báo một cách hợp lý, ta có thể ra phương án sớm để tăng tài nguyên khi hệ thống bị cao tải hoặc có thể biết được hệ thống đang bị tấn công hoặc bị lỗi để có kịch bản ứng cứu sớm. 