\begin{abstract}
    Ngoài việc lập trình thì vận hành và giám sát là khâu quan trọng trong vòng đời của một sản phẩn công nghệ (phần mềm). Vận hành một sản phẩm không hề dễ dàng hơn việc tạo ra nó. Khi một phần mềm được triển khai thực tế, lập trình viên cũng như người quản trị hệ thống luôn cần nắm rõ \textit{sức khoẻ} của sản phẩm ví dụ như phần mềm đang tiêu tốn bao nhiêu tài nguyên của máy tính, phần mềm xử lý yêu cầu nhanh hay chậm, hay có bao nhiêu lỗi xảy ra trong khung giờ... Những thông số hay độ đo này là thước đo để biết được sản phẩm có đủ tốt hoặc quan trọng hơn là cảnh báo sớm cho người làm phần mềm về những nguy cơ có thể xảy ra. Bài báo cáo này giới thiệu một loạt các kĩ thuật, độ đo, cách kết hợp những công cụ và việc vận dụng khéo léo giữa bản thân phần mềm (ghi log) và các công cụ đó. Trực quan hoá dữ liệu được sử dụng một cách triệt để nhằm tạo một cái nhìn từ tổng quan đến chi tiết giúp cho người vận hành hệ thống cũng như lập trình viên luôn giữ được phần mềm của mình trong tầm kiểm soát.
\end{abstract} 