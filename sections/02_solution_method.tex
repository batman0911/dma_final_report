\chapter{Phương pháp giải}
Tìm kiếm lân cận heuristics thường dựa trên các di chuyển lân cận thay đổi nhỏ so với lời giải hiện tại như là chuyển yêu cầu từ tuyến đường này sang tuyến đường khác hoặc là tráo đổi 2 yêu cầu, như trong bài của Nanry \& Barnes (2000) và Li \& Lim (2001). Những kiểu tìm kiếm lân cận này có thể kiểm tra một lượng lớn phương án trong thời gian ngắn, nhưng phương án được cập nhật rất nhỏ qua mỗi vòng lặp. Đó là điều khiến tác giả tin rằng các phương pháp này có thể gặp khó trong việc di chuyển từ vùng triển vọng này của phương án tới vùng khác khi phải đối mặt với các ràng buộc chặt ngay cả khi áp dụng metaheuristics.

Một trong những cách để giải quyết vấn đề này là cho phép tìm kiếm đi qua các phương án chấp nhận được bằng cách nới lỏng các ràng buộc (ví dụ trong Cordeau, Laporte \& Mercier (2001)). Tác giả chọn một cách tiếp cận khác, thay vì sử dụng bước đi chuyển nhỏ, tác giả sử dụng các bước di chuyển rất lớn mà có thể sắp xếp 30\% đến 40\% tất cả các yêu cầu trong một bước lắp. Sự đánh đổi của việc này là thời gian tính toán cần thiết và việc đánh giá các bước di chuyển trở nên lớn hơn khi so sánh với việc di chuyển các bước nhỏ. Số phương án được đánh giá bởi heuristic tiềm năng trên một đơn vị thời gian chỉ là một phần nhỏ của các phương án được đánh giá trên một heuristic chuẩn. Tuy nhiên, thuật toán cho hiệu năng rất tối trong các bài tests được trình bày trong \$4.

Heuristic tiềm năng dựa trên \textit{large neighbourhood search (LNS)}, được giới thiệu bởi Shaw (1997). LNS heuristic đã được áp dụng vào VRPTW với các kết quả tốt (Shaw 1997, 1998; Bent \& Van Hentenryck 2004). LNS heuristic tương tự với \textit{ruin and recreate} heuristic được trình bày bởi Schrimpf et al. (2000).

Mã giả cho một LNS heuristic tối thiểu được trình bày trong Algorithm 1

\begin{algorithm}
	\caption{PPO} 
	\begin{algorithmic}[1]
		\For {$iteration=1,2,\ldots$}
			\For {$actor=1,2,\ldots,N$}
				\State Run policy $\pi_{\theta_{old}}$ in environment for $T$ time steps
				\State Compute advantage estimates $\hat{A}_{1},\ldots,\hat{A}_{T}$
			\EndFor
			\State Optimize surrogate $L$ wrt. $\theta$, with $K$ epochs and minibatch size $M\leq NT$
			\State $\theta_{old}\leftarrow\theta$
		\EndFor
	\end{algorithmic} 
\end{algorithm}
Thuật toán giả định rằng lời giải ban đầu \textit{s} đã có ví dụ được tạo bằng một heuristic đơn giản. Tham số thứ 2 \textit{q} xác định phạm vi tìm kiếm. 

Dòng 5 và 6 của thuật toán là phần thú vị của heuristic. Ở dòng 5, một số yêu cầu được loại bỏ khỏi phương án hiện tại \textit{s'}, các yêu cầu lại được thêm vào ở dòng 6. Hiệu năng cũng như sự mạnh mẽ của heuristic phụ thuộc vào sự lựa chọn chiến thuật bỏ và thêm lại các yêu cầu. Trong các bài trước đó về LNS cho VRPTW và PDPTW (Shaw 1997; Bent and Van Hentenryck 2003a) các phương pháp \textit{gần tối ưu} được sử dụng để thêm lại các yêu cầu. Trong bài báo này, tác giả tiếp cận kiểu khác bằng cách sử dụng các cách mẹo thêm lại yêu cầu đơn giản. Mặc dù các cách thêm lại yêu cầu heuristic thường có chất lượng kém, nhưng chất lượng của LNS heurustic lại rất tốt, bởi vì các bước xấu được tạo ra bởi heuristic thêm lại yêu cầu dẫn đến sự đa dạng hóa hiệu quả của quá trình tìm kiếm. 

Phần còn lại của thuật toán cập nhật phương án tốt nhất (hiện tại) và tìm kiếm phương án mới (tốt hơn). Một tiêu chí chấp nhận đơn giản sẽ là chấp nhận tất cả các phương án cải tiến. Một tiêu chí như vậy đã được sử dụng trong các triển khai LNS trước đó (Shaw 1997).

Dòng 11 kiểm tra điều kiện dừng đã đạt được hay chưa. Trong code triển khai, tác giả dừng khi đạt một số vòng lặp nhất định.

Tham số $q \in \{0,...,n\}$ xác định kích cỡ tập lân cận. Nếu $q = 0$ thì có nghĩa là không có bước tìm kiếm nào hết vì không có yêu cầu nào được bỏ đi. Mặt khác nếu $q = n$, thì bài toán được giải luôn qua mỗi vòng lặp. Nói chung, $q$
càng lớn thì càng dễ di chuyển quanh không gian nghiệm, tuy nhiên khi $q$ lớn dần lên thì bước thêm lại yêu cầu sẽ chậm hơn. 

Tìm kiếm cục bộ LNS có thể được coi là một ví dụ về tìm kiếm lân cận quy mô rất lớn như được trình bày bởi Ahuja et al. (2002). Ahuja et al. định nghĩa các vùng lân cận có quy mô rất lớn là các vùng lân cận có kích thước tăng theo cấp số nhân như một hàm của kích thước bài toán hoặc các vùng lân cận đơn giản là quá lớn để tìm kiếm một cách rõ ràng trong thực tế. Tìm kiếm cục bộ LNS phù hợp với danh mục cuối cùng, vì có nhiều khả năng chọn yêu cầu để xóa và một số lượng lớn các lần thêm lại yêu cầu có thể. Một sự khác biệt quan trọng giữa heuristic được đề xuất và hầu hết các heuristic được mô tả trong Ahuja et al. (2002) là các heuristic sau thường kiểm tra một số lượng lớn các phương án trong khi heuristic được đề xuất trong bài này chỉ xem xét một số lượng tương đối nhỏ các phương án. 

Thay vì xem xét quá trình LNS như là một chuỗi hành động xoá và thêm lại, chúng ta có thể coi quá trình này là chuỗi hành động sửa lỗi và tối ưu. Trong thao tác sửa lỗi (\textit{fix}), một số lượng các phần tử của phương án hiện tại được sửa. Ví dụ, nếu phương án được biểu diễn dưới dạng một vector các biến, thao tác sửa lỗi có thể sửa một số biến này ở giá trị hiện tại của chúng. Sau đó, thao tác tối ưu hóa sẽ tối ưu hóa lại phương án trong khi vẫn tôn trọng việc sửa lỗi được thực hiện trong thao tác sửa lỗi trước đó. Góc nhìn này có thể giúp chúng ta áp dụng heuristic cho các bài toán mà các thao tác xóa-chèn có vẻ không trực quan. Trong phần \$3, tác giả giới thiệu \textit{adaptive large neighborhood search (ALNS)} để mô tả một thuật toán sử dụng một số vùng lân cận lớn theo cách thích ứng.
