
\chapter{LNS áp dụng cho PDPTW}
Phần này mô tả cách thuật toán LNS được ứng dụng vào bài toán PDPTW. LNS trong bài báo này khác với thuật toán được phát triển cho bài toán VRPTW và PDPTW được thực hiện bởi Shaw (1997, 1998), Bent và Van Hentenryck (2003a, 2004) ở một số điểm như sau:
\begin{enumerate}
    \item Chúng tôi sử dụng đồng thời nhiều phương thức heuristic thêm và xóa trong cùng 1 lần tìm kiếm, trong khi đó các phương pháp heuristic LNS trước đây sử dụng chỉ 1 phương thức thêm và 1 phương thức xóa. Phương thức heuristic xóa được trình bài trong Phần 3.1 và phương thức heuristic thêm được trình bài trong Phần 3.2. Phương thức lựa chọn được sử dụng cho subheuristic được mô tả trong Phần 3.3. Cơ chế được lựa chọn được xây dựng dựa trên số liệu thống kê thu thập được trong quá trình tìm kiếm đã được trình  bài trong Phần 3.4. Chúng tôi sử dụng thuật ngữ Adaptive Large Neighborhood Search (ALNS) heuristic khi áp dụng nhiều phương thức thêm và xóa cho thuật toán LNS và sử dụng dữ liệu thống kê trong quá trình tìm kiếm.
    \item Phương pháp heuristic đơn giản và nhanh được sử dụng để thêm các yêu cầu, trái ngược với phương pháp phân nhành và ràng buộc phức tạp của Shaw (1997, 1998) và Bent, VanHentenryck (2003a, 2004).
    \item Quá trình tìm kiếm được nhúng trong 1 metaheuristic mô phỏng khi mà các kỹ thuật LNS heuristic trước đó sử dụng các tiếp cận đơn giản. Điều này được trình bày trong Phần 3.5.
\end{enumerate}
Phần này cũng mô tả cách LNS heuristic có thể sử dụng trong 1 thuật toán đơn giản, dùng để tìm số lượng phương tiện nhỏ nhất cho 1 số lượng yêu cầu. Thuật toán tìm số lượng phương tiện nhỏ nhất chỉ hoạt động khi không giới hạn các phương tiện cùng loại.
\section{Xóa yêu cầu}
Có 3 phương pháp heuristic dùng cho việc xóa yêu được trình bày trong phần này. Cả 3 phương pháp đều sử dụng 1 cách và 1 số nguyên dương \textit{q} là đầu vào. Đầu ra là 1 phương án với \textit{q} là số yêu cầu đã được xóa. Hơn nữa, \textit{Shaw removal} heuristic và \textit{worst removal} heuristic có tham số \textit{p} biểu diễn cho mức độ ngẫu nhiên trong thuật toán.

\subsection{Phương pháp xóa Shaw Heuristic}
Phương pháp xóa này được phát triển bởi Shaw (1997, 1998). Cách trình bày trong phần này đã được chỉnh sửa lại để phù hợp với PDPTW. Ý tưởng chung là xóa bỏ các yêu cầu giống nhau, vì chúng ta hy vọng sẽ dễ dàng trộn các yêu cầu tương tự với nhau và tạo ra các phương án mới có thể tốt hơn. Nếu chúng ta chọn xóa bỏ các yêu cầu khác nhau, thì sau đó, việc thêm các yêu cầu mới sẽ không nhận lại được điều gì do các yêu cầu này có thể chỉ được thêm vào tại vị trí ban đầu của chúng hoặc ở các vị trí tồi tệ. Mức độ tương đồng giữa 2 yêu cầu \textit{i} và \textit{j} được định nghĩa dựa trên chỉ số độ tương đồng $R(i,j)$. Chỉ số này càng thấp thì 2 yêu cầu càng giống nhau.

Chỉ số độ tương đồng được sử dụng trong bài này bao gồm phụ thuộc vào 4 điều kiện: khoảng cách, thời gian, khối lượng và khả năng phương tiện có thể sử dụng để phục vụ 2 yêu cầu cùng lúc. Các điều kiện này được đánh trọng số và ký hiệu lần lượt là $\varphi$, $\chi$, $\psi$ và $\omega$. Chỉ số độ tương đồng được tính như sau:

\begin{equation}
\begin{split}
    R(i,j) = \varphi(d_{A(i), A(j)} + d_{B(i), B(j)}) + \chi(|T_{A(i)}-T_{A(j)}| + |T_{B(i)} - T_{B(j)}|)  + \\ \psi(||l_i - l_j) + \omega(1 - \frac{|K_i \cap K_j|}{min(|K_i|, K_j)})
\end{split}
\end{equation}

\textit{A(i)} và \textit{B(i)} biểu diễn cho điểm lấy và giao hàng của yêu cầu i, $T_i$ là thời gian khi đến địa điểm \textit{i}, $d_{ij}, l_{i}, K_i$ được định nghĩa trong Phần 1. Sử dụng biến quyết định $S_{ik}$ trong Phần 1, ta có $T_i = \sum_{k \in K} \sum_{j \in V_k} S_{ik} x_{ijk}$. Các trọng số $\varphi$, $\chi$, $\psi$ và $\omega$ lần lượt biểu diễn cho khoảng cách, sự kết nối tạm thời, nhu cầu khối tượng và sự bảo đảm 2 yêu cầu có độ liên quan cao nếu chỉ 1 hoặc không phương tiện nào có khả năng phục vụ cả 2 yêu cầu. Giả sử $d_{ij}, T_x, l_i$ được chuẩn hóa như sau: $0 \leqslant R_{i,j} \leqslant 2(\varphi + \chi) + \psi + \omega$. Điều này được đảm bảo khi $d_{ij}, T_x, l_i$ nằm trong khoảng [0, 1]. Chú ý rằng ta không thể tính toán \textit{R(i, j)} nếu yêu cầu \textit{i} hoặc \textit{j} được đặt trong hàng chờ.

Mức độ liên quan được sử dụng để loại bỏ các yêu cầu trong cùng 1 cung đường được mô tả bởi Shaw (1997), được mô tả trong Algorithm 2. Ban đầu, 1 yêu cầu được chọn ngẫu nhiên và xóa. Trong các vòng lặp tiếp theo, thuật toán sẽ thực hiện xóa các yêu cầu giống với các yêu cầu đã được xóa. Tham số $p \geqslant 1$ biểu diễn cho sự ngẫu nhiên trong cách lựa chọn yêu cầu (\textit{p} càng thấp thì độ ngẫu nhiên càng cao).

\begin{algorithm}
	\caption{Shaw Removal} 
	\begin{algorithmic}[1]
        \Require $s \in {solutions}, q \in \mathbb{N}, p \in \mathbb{R}_{+}$
        \State request: r = a randomly selected request from s;
        \State set of requests: $\mathbb{D}=\{r\}$;
        \While {$|\mathbb{D}| < q$}
		  \State r = a randomly selected request from $\mathbb{D}$;
            \State Array: L = an array containing all request from s not in $\mathbb{D}$;
            \State sort \textit{L} such that $i<j \Rightarrow R(r, L\left[ i \right]) < R(r, L\left[ j \right])$;
            \State choose a random number \textit{y} from the interval [0, 1);
            \State $\mathbb{D}=\mathbb{D}\cup {L \left[ y^p|L| \right]}$
        \EndWhile
    \State remove the requests in $\mathbb{D}$ from \textit{s};
	\end{algorithmic} 
\end{algorithm}

Lưu ý rằng việc sắp xếp tại dòng 6 có thể lược bỏ được trong quá trình triển khai thuật toán thực tế vì chỉ cần sử dụng thuật toán chọn thời gian tuyến tính (Cormen và nhóm tác giả 2001) trong dòng 8 là đủ.

\subsection{Phương pháp xóa ngẫu nhiên}
Thuật toán xóa ngẫu nhiên đơn giản chọn ngẫu nhiên \textit{q} yêu cầu và loại bỏ chúng từ tập phương án. Kỹ thuật này có thể coi là 1 trường hợp đặc biệt của phương pháp xóa Shaw với \textit{p}=1. Tuy nhiên, tác giả đã cài đặt 1 kỹ thuật riêng biệt nhanh hơn.

\subsection{Phương pháp xóa tệ nhất}
Cho 1 yêu cầu \textit{i} được phục vụ bởi vài phương tiện trong tập phương án \textit{s}, tác giả định nghĩa chi phí của yêu cầu \textit{cost} như sau: $cost(i,s)=f(s)-f_{-i}(s)$ với $f_{-i}(s)$ là chi phí của phương án mà không có yêu cầu i (yêu cầu được xóa mà không chuyển đến hàng chờ). Nó có vẻ hợp lý khi cố gắng xóa các yêu cầu với chi phí cao và thêm chúng vào vị trí khác trong tập phương án để có được phương án tốt hơn. Vì vậy, tác giả đề xuất 1 kỹ thuật xóa các yêu cầu có chi phí $cost(i, s)$ cao.

Phương pháp này được biểu diễn trong Algorithm 3. Nó sử dụng lại vài ý tưởng từ Phần 3.1.1.

Lưu ý rằng việc thực hiện xóa là ngẫu nhiên, với mức độ ngẫu nhiên được kiểm soát bởi biến \textit{p} trình bày trong Phần 3.1.1. Điều này được thực hiện để tránh trường hợp yêu cầu bị xóa bỏ lặp đi lặp lại.

\begin{algorithm}
    \caption{Worst Removal} 
	\begin{algorithmic}[1]
        \Require $s \in {solutions}, q \in \mathbb{N}, p \in \mathbb{R}_{+}$
        \While {$q > 0$}
		  \State Array: \textit{L} = All planned requests \textit{i}, sorted by descending \textit{cost(i,s)};
            \State choose a random number \textit{y} in the interval $[0, 1)$;
            \State request: $r = L\left[ y^p |L| \right]$;
            \State remove r from solution s;
            \State $q = q-1$;
        \EndWhile
	\end{algorithmic} 
\end{algorithm}

Có thể nói rằng phương pháp xóa Shaw và phương pháp xóa tệ nhất thuộc về 2 trường phái khác nhau. Phương pháp của Shaw thiên về chọn các yêu cầu dễ dàng trao đổi, trong khi đó phương pháp xóa tệ nhất lựa chọn yêu cầu dường như được đặt vào vị trí sai trong tập phương án.
\section{Các phương pháp thêm yêu cầu}
Các kỹ thuật thêm cho các bài toán định tuyến thường được chia thành 2 dạng: tuần tự và song song. Điểm khác biệt giữa 2 dạng này là phương pháp tuần tự xây dựng 1 tuyến đường chỉ trong 1 thời điểm còn phương pháp song song xây dựng các tuyến đường cùng lúc. 2 phương pháp này được trình bày chi tiết trong Potvin and Rousseau (1993). Kỹ thuật được trình bày trong bài này là kỹ thuật song song. Người đọc nên quan sát rằng kỹ thuật thêm được trình bày ở đây được sử dụng trong bối cảnh mà chúng được cung cấp 1 số tuyến đường 1 phần và 1 số yêu cầu để chèn hơn là việc xây dựng thuật toán từ đầu.

\subsection{Phương pháp heuristic tham lam cơ bản}
Heuristic tham lam cơ bản là 1 kỹ thuật xây dựng đơn giản. Nó thực hiện tối đa \textit{n} lần lặp vì nó chèn thêm 1 yêu cầu trong mỗi lần lặp. Với $\Delta f_{i, k}$ biểu diễn cho sự thay đổi trong giá trị phát sinh của mục tiêu bằng cách chèn thêm yêu cầu \textit{i} vào tuyến đường \textit{k} tại vị trí mà giá trị mục tiêu là nhỏ nhất. Nếu không chèn yêu cầu \textit{i} và tuyến đường \textit{k} thì ta đặt $\Delta f_{i, k} = \infty$ và $c_i = min_{k \in K}\{\Delta f_{i, k}\}$ . Nói cách khác, $c_i$ là chi phí khi chèn thêm yêu cầu \textit{i} vào vị trí tốt nhất trong giải pháp, gọi là vị trí chi phí nhỏ nhất. Cuối cùng, ta chọn yêu cầu \textit{i} với: $min_{i \in U} c_i$ và chèn nó vào vị trí chi phí nhỏ nhất. \textit{U} là tập các yêu cầu chưa được lên kế hoạch. Quá trình này tiếp tục cho đến khi tất cả các yêu cầu được thêm hoặc không còn yêu cầu nào nữa.

Nhận thấy trong mỗi vòng lặp, thuật toán chỉ thay đổi 1 tuyến đường (tuyến mà yêu cầu mới được thêm vào), và thuật toán không cần phải tính toán lại chi phí chèn thêm trong tất cả các tuyến đường khác. Điểm này giúp tăng hiệu năng cho thuật toán.

Dễ dàng nhận thấy 1 vấn đề với cách tiếp cận này là nó thường trì hoãn việc đặt các yêu cầu có chi phí cao cho các lần lặp cuối cùng, nơi chúng ta không có nhiều cơ hội cho việc chèn thêm yêu cầu vì nhiều tuyến đường đều đã kín. Phương pháp heuristic được trình bày trong phần sau sẽ cố gắng giải quyết hạn chế này.

\subsection{Phương pháp regret heuristic}
Phương pháp này cố gắng cải thiện nhược điểm của kỹ thuật tham lam bằng cách kiểm tra lại kết quả sau khi chọn chèn thêm yêu cầu. Đặt $x_{ik} \in {1, ..., m}$ là biến biểu diễn tuyến đường cho yêu cầu \textit{i} có chi phí chèn thêm vào thấp thứ \textit{k}, điều này có nghĩa là $\Delta f_{i, x_{ik}} \leqslant \Delta f_{i, x_{ik'}}$. Sử dụng ký hiệu này, ta có thể biểu diễn $c_i$ từ Phần 3.2.1 như sau: $c_i = \Delta f_{i, x_{i1}}$. Trong phương pháp này, chúng ta có thể định nghĩa 1 giá trị \textit{regret} $c_i^* = \Delta f_{i, x_{i2}} - \Delta f_{i, x_{i1}}$ . Nói cách khác, giá trị regret là khoảng cách giữa chi phí của việc chèn thêm yêu cầu vào tuyến đường tốt nhất so với tuyến đường tốt thứ 2 tốt thứ 2. Trong mỗi vòng lặp, thuật toán chọn ra yêu cầu \textit{i} thỏa mãn điều kiện: $max_{i \in U} \{C_i^*\}$. Với điều kiện này, yêu cầu được đảm bảo thêm vào vị trí có chi phí nhỏ nhất. Điều này khiến cho các ràng buộc bị phá vỡ. Nói cách khác, thuật toán sẽ thực hiện chèn yêu cầu vào giải pháp nếu không sẽ hối tiếc khi không thực hiện điều này.

Phương pháp này có thể mở rộng 1 cách tự nhiên để định nghĩa 1 lớp các phương pháp regret heuristic: phương pháp k-regret heuristic là 1 phương pháp mà mỗi lần thêm yêu cầu vào giải pháp cần phải thỏa mãn điều kiện:
\begin{equation}
    \max\limits_{i \in U} \{ \sum_{j=1}^k (\Delta f_{i, x_{ij}} - \Delta f_{i, x_{i1}}) \}
\end{equation}
Nếu 1 vài yêu cầu không thể được chèn thêm vào ít nhất $m-k+1$ tuyến đường thì yêu cầu đó sẽ được chèn vào số lượng tuyến đường ít nhất (có thể là 1). Các ràng buộc bị phá vỡ bởi việc lựa chọn yêu cầu với chi phí chèn tốt nhất. Yêu cầu được chèn vào vị trí ít chi phí nhất. Thuật toán heuristic hối tiếc được trình bày trong Phần 3.2.1 là 1-regret heuristic bởi vì cách phá vỡ quy tắc. Nói cách khác, với k>2 thì thuật toán sẽ tính toán chi phí của việc thêm vào 1 yêu cầu qua k tuyến đường tốt nhất và chèn yêu cầu mà khoảng cách chi phí giữa việc thêm nó vào tuyến đường tốt nhất với tuyến đường tốt thứ k-1 là lớn nhất. So sánh với 2-regret heuristic, thuật toán với giá trị lớn của \textit{k} khám phá ra sớm hơn khả năng bị giới hạn khi thêm 1 yêu cầu.

Heuristic hối tiếc đã được sử dụng bởi Potvin và Rousseau (1993) cho VRPTW. Kỹ thuật trong bài báo của họ có thể được phân loại vào k-regert heuristic với k=m, vì tất cả các tuyến đường được xem xét trong 1 biểu thức tương tự (3.2). Nhóm tác giả không sử dụng sự thay đổi trong giá trị mục tiêu để đánh giá chi phí của 1 lần chèn mà sử dụng 1 hàm chi phí đặc biệt. Các reget heuristic cũng có thể được sử dụng cho các vấn đề tối ưu hóa tổ hợp bên ngoài miền định tuyến phương tiện, một ví dụ về ứng dụng cho vấn đề chuyển nhượng tổng quát được trình bày bởi Martello và Toth (1981). Như trong phần trước, có 1 sự thật là chỉ 1 tuyến đường được thay đổi trong mỗi vòng lặp để tăng tốc kỹ thuật regret heuristic.
\section{Lựa chọn phương pháp xóa và chèn}
Trong Phần 3.1, tác giả đã định nghĩa 3 phương pháp xóa (Shaw, ngẫu nhiên, tệ nhất) và trong Phần 3.2 đã định nghĩa 1 lớp các insertion heuristic. Ta có thể lựa chọn kết hợp 1 phương pháp xóa và 1 phương pháp thêm với nhau để phục vụ quá trình tìm kiếm, nhưng trong bài báo này, tác giả đề xuất sử dụng tất cả các phương pháp. Lý do cho việc này là reget-2 heuristic có thể phù hợp với trường hợp này nhưng reget-4 lại là lựa chọn tốt nhất cho trường hợp khác. Chúng tối tin rằng, sự thay thế giữa các phương pháp thêm và xóa sẽ đảm bảo cho sự bền vững và hiệu quả.

Để lựa chọn phương pháp heuristic để sử dụng, tác giả gán trọng số cho các heuristic khác nhau và sử dụng một nguyên tắc bánh xe lựa chọn. Nếu có \textit{k} heuristic với trong số $\omega_i, i \in \{1,2,...,k\}$, ta chọn heuristic \textit{j} với xác suất:
\begin{equation}
  \frac{\omega_j}{\sum_{i=1}^k \omega_i}
  \label{eq:20}
\end{equation}
Lưu ý rằng, thuật toán chèn được chọn độc lập với thuật toán xóa và ngược lại. Các trọng số này có thể được thiết lập bằng tay hoặc nó có thể là 1 quá trình phức tạp nếu sử dụng nhiều phương pháp xóa và chèn. Thay vào đó, 1 thuật toán điều chỉnh trọng số được trình bày trong Phần 3.4.

\section{Điều chỉnh trọng số tự động}
Phần này trình bày cách các trọng số $\omega_i$ được giới thiệu trong Phần 3.3 có thể tự động điều chỉnh khi sử dụng phương pháp thống kê từ các vòng lặp trước. Ý tưởng chung là theo dõi 1 điểm số đại diện cho độ hiệu quả của thuật toán trong các vòng lặp gần đây. Điểm số này càng cao thì thuật toán càng hiệu quả. Cả quá trình tìm kiếm được chia thành nhiều bước. 1 bước là 1 số vòng lặp của ALNS heuristic; ở đây chúng tôi chọn 1 bước là 100 vòng lặp. Điểm số heuristic được đặt bằng 0 khi bắt đầu mỗi bước và được tăng thêm $\sigma_1, \sigma_2, \sigma_3$ tùy các trường hợp ghi trong Bảng 1.

% \begin{table}[]
%     \centering
%     \begin{tabular}{l|l}
%         \hline
%         Tham số     &   Mô tả
%         \hline
%         $\sigma_1$  &   Hoạt động xóa-chèn cuối cùng dẫn đến một giải pháp mới tốt nhất trong phạm vi toàn cục.
%         $\sigma_2$  &   Thao tác xóa-chèn cuối cùng dẫn đến một giải pháp chưa được chấp nhận trước đó. Chi phí của giải pháp mới tốt hơn chi phí của giải pháp hiện tại.
%         $\sigma_3$  &   Thao tác xóa-chèn cuối cùng dẫn đến một giải pháp chưa được chấp nhận trước đó. Chi phí của giải pháp mới tồi tệ hơn chi phí của giải pháp hiện tại, nhưng giải pháp đã được chấp nhận.
%         \hline
%     \end{tabular}
%     \caption{Caption}
% \end{table}

Trong mỗi vòng lặp, chúng tôi áp dụng 2 heuristic: xóa và chèn. Các điểm số cho cả 2 phương pháp heuristic được cập nhật cùng 1 lượng, vì ta không thể biết chắc chắn sự tốt lên của thuật toán là do phương thức xóa hay chèn.

Tại mỗi lần kết thúc 1 bước, ta tính toán lại trọng số mới sử dụng các điểm số cũ. Cho $\omega_{ij}$ là trọng số của heuristic \textit{i} được sử dụng tại bước \textit{j} với trọng số như trong công thức 3.3. Trong bước đầu tiên, ta đánh trọng số các heuristic bằng nhau. Sau đó, khi bước \textit{j} kết thúc, ta tính toán trọng số cho tất cả heuristic \textit{i} để sử dụng cho bước thứ \textit{j} + 1 như sau:
\begin{equation}
    \omega_{i, j+1} = \omega_{ij}(1-r)+r\frac{\pi_i}{\theta_i}
\end{equation}
Trong đó, $\pi_i$ là điểm số của heuristic \textit{i} được nhận trong bước cuối cùng, $\theta_i$ là số lần ta cố gắng sử dụng heuristic \textit{i} trong bước thực hiện đó. \textit{Reaction factor - r} điều khiển cách trọng số điều chỉnh nhanh hay chậm, phản ứng với sự thay đổi của độ hiệu quả của thuật toán. Nếu \textit{r} = 0 thì ta không sử dụng điểm và các trọng số sẽ không đổi. Nếu \textit{r} = 1 thì điểm đánh giá được nhận trong lần thực hiện cuối cùng sẽ quyết định trọng số.

% Vị trí hình 1

Hình 1 cho thấy cách các trọng số của 3 heuristic xóa biến đổi theo thời gian khi giải quyết 1 bài toán. Biểu đồ giảm dần vì các tiêu chí mô phỏng chỉ chấp nhận các nước di tốt, do đó heuristic khó đạt điểm cao.

\section{Tiêu chí chấp nhận và dừng}
Như đã trình bày trong phần 2, 1 tiêu chí đơn giản có thể dùng là chỉ chấp nhận các phương án tốt hơn hiện tại. Điều này cho chúng ta 1 phương pháp descent heuristic giống như phương pháp đã trình bày bởi Shaw(1997). Tuy nhiên, mỗi heuristic có 1 khuynh hướng bị mắc kẹt trong 1 tối ưu cục bộ, nên có vẻ hợp lý khi chấp nhận các phương án tồi tệ hơn phương án hiện tại. Để làm điều này, tác giả sử dụng tiêu chí chấp nhận từ mô phỏng luyện kim (simulated annealing). Nghĩa là ta chấp nhận 1 phương án $s'$ cho phương án hiện tại $s$ với xác suất $e^{-(f(s')-f(s))/T}$ với \textit{T} > 0 là nhiệt độ (\textit{temperature}).

Nhiệt độ được bắt đầu từ $T_{start}$ và giảm ở mỗi vòng lặp theo biểu thức $T=T \cdot c$, trong đó $0<c<1$ là tỷ lệ làm lạnh \textit{cooling rate}. 1 lựa chọn tốt của $T_{start}$ là độc lập với bài toán thực hiện, do đó thay vì xác định rõ $T_{start}$ như là 1 tham số, ta tính toán nó bằng cách theo dõi phương án ban đầu (\textit{initial solution}). Đầu tiên chi phí $z'$ của phương án được tính toán dùng 1 hàm mục tiêu đã được sửa đổi. Trong hàm mục tiêu này, $\gamma$ (chi phí của việc có yêu cầu trong hàng chờ) được đặt là 0. Nhiệt độ bắt đầu lúc này được đặt sao cho phương án kém hơn $\omega\%$ phương án hiện tại được chấp nhận với xác suất là 0.5. Lý do cho việc đặt $\gamma$ là 0 vì tham số này thường lớn và có thể gây ra việc nhiệt độ ban đầu quá lớn nếu phương án ban đầu có vài yêu cầu trong hàng chờ. $\omega$ hiện tại là 1 tham số cần được khởi tạo. Ta gọi tham số này là tham số điều khiển nhiệt độ bắt đầu (\textit{start temperature control parameter}). Thuât toán dừng lại khi 1 số vòng lặp nào đó của LNS được thực thi xong.

\section{Áp dụng nhiễu vào hàm mục tiêu}
Như thuật toán chèn đã được trình bày khá nông và còn thiếu sót, tác giả tin rằng nên thêm yếu tố ngẫu nhiên vào heuristic chèn sao cho không phải lúc nào cũng thực hiện chuyển động tốt nhất trong phần tối ưu cục bộ. Điều này đạt được bằng cách thêm nhiễu vào hàm mục tiêu. Mỗi lần tính toàn chi phí $C$ của 1 vòng lặp của 1 yêu cầu trong tuyến đường, ta cũng tính 1 số ngẫu nhiên nhiễu (noise) có giá trị trong khoảng $\left[ -maxN, maxN \right]$ và tính chi phí thêm $C'=max\{ 0, C+noise \}$. Tại mỗi vòng lặp ta quyết định nếu sử dụng \textit{C} hay \textit{C'} để làm rõ tính hiệu quả của vòng lặp. Quyết định này được thực hiện bởi cơ chế thích nghi được mô tả ở phần trước bằng cách theo dõi cách mà nhiều thường tác động lên việc chèn và các lần chèn hiệu quả.

Để tạo ra số lượng nhiễu liên quan đến các thuộc tính của bài toán, ta tính toán $maxN = \eta\cdot max_{i,j \in V} \{d_{ij}\}$, trong đó $\eta$ là tham số điều khiển lượng nhiễu. Ta chọn $maxN$ độc lập với khoảng cách $d_{ij}$ là 1 phần quan trọng của mục tiêu trong tất cả các bài toán được xem xét trong paper này.

Có vẻ như không cần thiết để thêm nhiễu vào insertion heuristic, vì kỹ thuật này được sử dụng trong mô phỏng luyện kim vì nó đã có sẵn sự ngẫu nhiên. Tuy nhiên, tác giả tin rằng sự ứng dụng của nhiễu quan trọng vì vùng lân cận được tìm kiếm bởi trung bình của các heuristic chèn và không được lấy mẫu ngẫu nhiên. Nếu không có các ứng dụng của nhiễu, ta không nhận được đầy đủ lợi ích của siêu giả lập luyện kim. Phỏng đoán này được hỗ trợ bởi các thí nghiệm tính toán trong Bảng 4.

\section{Cực tiểu số lượng phương tiện sử dụng}
Cực tiểu hóa lượng xe sử dụng dùng để phục vụ tất cả các yêu cầu thường có được sự ưu tiên cao nhất trong lý thuyết định tuyến phương tiện. Các phương pháp heuristic được trình bày đến lúc này đều không có khả năng đối phó với mục tiêu như vậy. Nhưng bằng cách sử dụng thuật toán 2 giai đoạn đơn giản giúp tối thiểu các phương tiện ở giai đoạn đầu và tối thiểu mục tiêu thứ 2 (thường là khoảng cách di chuyển) trong giai đoạn 2, ta có thể xử lý chúng. Thuật toán cực tiểu hóa phương tiện chỉ hoạt động với khi các phương tiện là đồng nhất. Ta có thể giả định rằng số lượng xe có sẵn là vô hạn, do đó việc xây dựng 1 giải pháp khả thi ban đầu luôn có thể thực hiện được. Một phương pháp 2 giai đoạn được sử dụng bởi Bent và Van Hentenryck (2003a, 2004), nhưng trong khi họ sử dụng vùng lân cận khác và siêu heuristic cho 2 giai đoạn, tác giả sử dụng cùng 1 heuristic cho cả 2 giai đoạn.

Giai đoạn cực tiểu hóa số lượng phương tiện được thực hiện như sau: Đầu tiên, 1 giải pháp có thể thực hiện được khởi tạo sử dụng 1 chuỗi phương thức chèn giúp xây dựng 1 tuyến đường trong 1 thời điểm cho đến khi tất cả yêu cầu được lên kế hoạch. Số lượng phương tiện được sử dụng trong giải pháp này là số  phương tiện cần thiết ước lượng ban đầu. Bước tiếp theo là loại bỏ 1 tuyến đường từ giải pháp có thể thực hiện được. Các yêu cầu được đặt trên tuyến đường bị loại bỏ được đưa vào hàng chờ. Bài toán sau đó được giải quyết bởi LNS heuristic. Khi heuristic chạy, 1 giá trị cao được gán cho $\gamma$, những yêu cầu được đưa ra khỏi hàng chờ nếu có thể. Nếu heuristic có khả năng tìm giải pháp có thể phục vụ tất cả yêu cầu, 1 ứng viên mới cho cực tiểu số lượng phương tiện đã được tìm thấy. Khi đó, LNS heuristic dừng ngay lập tức, 1 tuyến đường nữa được loại bỏ khỏi giải pháp và quá trình được lặp lại. Nếu LNS heuristic dừng lại khi không tìm được giải pháp có thể phục vụ hết các yêu cầu thì thuật toán lùi lại giải pháp trước đó mà có thể phục vụ hết tất cả các yêu cầu. Giải pháp này được sử dụng như là 1 giải pháp khởi tạo của giai đoạn 2 của thuật toán, khi mà chỉ đơn giản áp dụng LNS thông thường.

Để giảm thời gian chạy của giai đoạn tối thiểu số lượng phương tiện, giai đoạn này chỉ cho phép dành $\phi$ vòng lặp LNS cùng nhau, do đó nếu ứng dụng đầu tiên của heuristic LNS, giả sử $\alpha$ vòng lặp để tìm ra giải pháp với tất cả yêu cầu được lên kế hoạch, thì giai đoạn cực tiểu hóa phương tiện chỉ được thực hiện trong $\phi - \alpha$ vòng lặp LNS. 1 cách khác để giới hạn thời gian chạy là dừng heuristic LNS khi nó có vẻ không tìm được giải pháp nào có khả năng phục vụ hết các yêu cầu. Trong thực tế, điều này được cài đặt bởi việc dừng LNS nếu 5 hoặc nhiều hơn yêu cầu không được lên kế hoạch được tìm ra trong vòng lặp LNS cuối cùng $\tau$. Trong thí nghiệm tính toán $\phi$ được đặt là 25,000 và $\tau$ là 2,000.

\section{Thảo luận}
Sử dụng một số heuristic chèn và xóa trong quá trình tìm kiếm có thể được coi là sử dụng tìm kiếm cục bộ với một số vùng lân cận. Theo hiểu biết tốt nhất của tác giả, ý tưởng này chưa từng được sử dụng trong tài liệu LNS trước đây. Tìm kiếm Vùng lân cận Biến đổi (VNS) có liên quan được đề xuất bởi Mladenovi´c và Hansen (1997). VNS là một khung siêu dữ liệu sử dụng một họ các vùng lân cận được tham số hóa. Metaheuristic đã nhận được khá nhiều sự chú ý trong những năm gần đây và đã mang lại những kết quả ấn tượng cho nhiều vấn đề. Khi ALNS sử dụng một số vùng lân cận không liên quan, VNS thường dựa trên một vùng lân cận duy nhất được tìm kiếm với độ sâu thay đổi.

Một số metaheuristic có thể được sử dụng ở cấp cao nhất của ALNS để giúp heuristic thoát khỏi tối thiểu cục bộ. tác giả đã chọn sử dụng mô phỏng luyện kim vì heuristic ALNS đã chứa phần tử lấy mẫu ngẫu nhiên. Để thảo luận thêm về các khuôn khổ metaheuristic được sử dụng liên quan đến ALNS, hãy xem bài báo tiếp theo (Pisinger và Ropke 2005).

Hàng chờ yêu cầu là một thực thể có ý nghĩa đối với nhiều ứng dụng thực tế. Trong các vấn đề được xem xét trong Phần 4, tác giả không chấp nhận các giải pháp có yêu cầu đột xuất, nhưng hàng chờ yêu cầu cho phép tác giả truy cập các giải pháp không khả thi trong giai đoạn chuyển tiếp, cải thiện tìm kiếm tổng thể. Hàng chờ yêu cầu đặc biệt quan trọng khi giảm thiểu số lượng phương tiện.


