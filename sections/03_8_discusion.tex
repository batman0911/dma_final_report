\section{Thảo luận}
Sử dụng một số heuristic chèn và xóa trong quá trình tìm kiếm có thể được coi là sử dụng tìm kiếm cục bộ với một số vùng lân cận. Theo hiểu biết tốt nhất của chúng tôi, ý tưởng này chưa từng được sử dụng trong tài liệu LNS trước đây. Tìm kiếm Vùng lân cận Biến đổi (VNS) có liên quan được đề xuất bởi Mladenovi´c và Hansen (1997). VNS là một khung siêu dữ liệu sử dụng một họ các vùng lân cận được tham số hóa. Metaheuristic đã nhận được khá nhiều sự chú ý trong những năm gần đây và đã mang lại những kết quả ấn tượng cho nhiều vấn đề. Khi ALNS sử dụng một số vùng lân cận không liên quan, VNS thường dựa trên một vùng lân cận duy nhất được tìm kiếm với độ sâu thay đổi.

Một số metaheuristic có thể được sử dụng ở cấp cao nhất của ALNS để giúp heuristic thoát khỏi tối thiểu cục bộ. Chúng tôi đã chọn sử dụng mô phỏng luyện kim vì heuristic ALNS đã chứa phần tử lấy mẫu ngẫu nhiên. Để thảo luận thêm về các khuôn khổ metaheuristic được sử dụng liên quan đến ALNS, hãy xem bài báo tiếp theo (Pisinger và Ropke 2005).

Hàng chờ yêu cầu là một thực thể có ý nghĩa đối với nhiều ứng dụng thực tế. Trong các vấn đề được xem xét trong Phần 4, chúng tôi không chấp nhận các giải pháp có yêu cầu đột xuất, nhưng hàng chờ yêu cầu cho phép chúng tôi truy cập các giải pháp không khả thi trong giai đoạn chuyển tiếp, cải thiện tìm kiếm tổng thể. Hàng chờ yêu cầu đặc biệt quan trọng khi giảm thiểu số lượng phương tiện.
