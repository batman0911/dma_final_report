\section{Cực tiểu số lượng phương tiện sử dụng}
Cực tiểu hóa lượng xe sử dụng dùng để phục vụ tất cả các yêu cầu thường có được sự ưu tiên cao nhất trong lý thuyết định tuyến phương tiện. Các phương pháp heuristic được trình bày đến lúc này đều không có khả năng đối phó với mục tiêu như vậy. Nhưng bằng cách sử dụng thuật toán 2 giai đoạn đơn giản giúp tối thiểu các phương tiện ở giai đoạn đầu và tối thiểu mục tiêu thứ 2 (thường là khoảng cách di chuyển) trong giai đoạn 2, ta có thể xử lý chúng. Thuật toán cực tiểu hóa phương tiện chỉ hoạt động với khi các phương tiện là đồng nhất. Ta có thể giả định rằng số lượng xe có sẵn là vô hạn, do đó việc xây dựng 1 phương án khả thi ban đầu luôn có thể thực hiện được. Một phương pháp 2 giai đoạn được sử dụng bởi Bent và Van Hentenryck (2003a, 2004), nhưng trong khi họ sử dụng vùng lân cận khác và siêu heuristic cho 2 giai đoạn, tác giả sử dụng cùng 1 heuristic cho cả 2 giai đoạn.

Giai đoạn cực tiểu hóa số lượng phương tiện được thực hiện như sau: Đầu tiên, 1 phương án có thể thực hiện được khởi tạo sử dụng 1 chuỗi phương thức chèn giúp xây dựng 1 tuyến đường trong 1 thời điểm cho đến khi tất cả yêu cầu được lên kế hoạch. Số lượng phương tiện được sử dụng trong phương án này là số  phương tiện cần thiết ước lượng ban đầu. Bước tiếp theo là loại bỏ 1 tuyến đường từ phương án có thể thực hiện được. Các yêu cầu được đặt trên tuyến đường bị loại bỏ được đưa vào hàng chờ. Bài toán sau đó được giải quyết bởi LNS heuristic. Khi heuristic chạy, 1 giá trị cao được gán cho $\gamma$, những yêu cầu được đưa ra khỏi hàng chờ nếu có thể. Nếu heuristic có khả năng tìm phương án có thể phục vụ tất cả yêu cầu, 1 ứng viên mới cho cực tiểu số lượng phương tiện đã được tìm thấy. Khi đó, LNS heuristic dừng ngay lập tức, 1 tuyến đường nữa được loại bỏ khỏi phương án và quá trình được lặp lại. Nếu LNS heuristic dừng lại khi không tìm được phương án có thể phục vụ hết các yêu cầu thì thuật toán lùi lại phương án trước đó mà có thể phục vụ hết tất cả các yêu cầu. phương án này được sử dụng như là 1 phương án khởi tạo của giai đoạn 2 của thuật toán, khi mà chỉ đơn giản áp dụng LNS thông thường.

Để giảm thời gian chạy của giai đoạn tối thiểu số lượng phương tiện, giai đoạn này chỉ cho phép dành $\phi$ vòng lặp LNS cùng nhau, do đó nếu ứng dụng đầu tiên của heuristic LNS, giả sử $\alpha$ vòng lặp để tìm ra phương án với tất cả yêu cầu được lên kế hoạch, thì giai đoạn cực tiểu hóa phương tiện chỉ được thực hiện trong $\phi - \alpha$ vòng lặp LNS. 1 cách khác để giới hạn thời gian chạy là dừng heuristic LNS khi nó có vẻ không tìm được phương án nào có khả năng phục vụ hết các yêu cầu. Trong thực tế, điều này được cài đặt bởi việc dừng LNS nếu 5 hoặc nhiều hơn yêu cầu không được lên kế hoạch được tìm ra trong vòng lặp LNS cuối cùng $\tau$. Trong thí nghiệm tính toán $\phi$ được đặt là 25,000 và $\tau$ là 2,000.
