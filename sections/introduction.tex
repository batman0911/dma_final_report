\chapter*{Giới thiệu}
Trong các biến thể của bài toán lấy và giao hàng với ràng buộc thời gian (PDPTW), tác giả được cung cấp một số lượng yêu cầu và phương tiện. Một yêu cầu về việc lấy hàng tại một địa điểm và giao chúng đến một địa điểm khác. Hai khung giờ ràng buộc được chỉ định cho mỗi yêu cầu: khung giờ lấy hàng chỉ định thời điểm có thể lấy hàng và khung giờ giao hàng cho biết khi nào hàng hóa có thể được giao. Thêm vào đó, \textit{thời gian phục vụ} (service time) được gắn với mỗi lần lấy và giao hàng. Thời gian phục vụ cho biết sẽ mất bao lâu để thực hiện việc lấy hàng hoặc giao hàng. Một xe được phép đến địa điểm trước khi bắt đầu khung giờ, nhưng chiếc xe đó phải đợi cho đến đúng lúc khung giờ bắt đầu mới có thể bắt đầu hoạt động. Một xe có thể không bao giờ đến địa điểm sau khi kết thúc ràng buộc thời gian của địa điểm đó.

Mỗi yêu cầu được chỉ định một tập hợp các xe khả thi. Điều này có thể được sử dụng để mô hình hóa các tình huống, ví dụ như khi một số phương tiện không thể đi vào một địa điểm nhất định do kích thước của phương tiện.

Mỗi xe có một sức chứa giới hạn, và nó bắt đầu và kết thúc nhiệm vụ của mình tại các địa điểm nhất định được gọi là bến đầu và bến cuối. Điểm xuất phát và kết thúc không nhất thiết phải giống nhau và hai phương tiện có thể có bến đầu và bến cuối khác nhau. Hơn nữa, mỗi chiếc xe được chỉ định một thời gian bắt đầu và kết thúc. Thời gian bắt đầu cho biết thời điểm xe phải rời khỏi vị trí bắt đầu và thời gian kết thúc biểu thị thời gian muộn nhất phải đến được địa điểm cuối cùng. Lưu ý rằng xe phải rời khỏi kho vào thời gian bắt đầu được chỉ định, mặc dù điều này có thể tạo ra khoảng thời gian chờ tại nơi xuất phát.

Nhiệm vụ của tác giả là xây dựng những tuyến đường hợp lệ cho các xe. Một tuyến đường là hợp lệ nếu ràng buộc thời gian và các giới hạn về sức chứa được tuân thủ, mỗi lần lấy hàng được thực hiện trước lần giao hàng tương ứng, việc lấy và giao hàng tương ứng được thực hiện trên cùng một tuyến đường và phương tiện chỉ được phép phục vụ các yêu cầu được giao. Các tuyến đường nên được xây dựng sao cho chúng giảm thiểu \textit{hàm chi phí} (cost function) được mô tả dưới đây.

Vì số lượng xe có hạn, tác giả có thể gặp phải tình huống không thể chỉ định một số yêu cầu cho xe. Những yêu cầu này được đặt trong một \textit{ngân hàng yêu cầu} (request bank) ảo. Trong thực tế, người điều hành là người quyết định phải làm gì với những yêu cầu như vậy. Ví dụ, nhà điều hành có thể quyết định thuê thêm xe để phục vụ các yêu cầu còn lại.

Mục tiêu của bài toán là cực tiểu hóa tổng trọng số gồm ba thành phần sau: (1) tổng quãng đường các xe đã đi, (2) tổng thời gian mỗi xe đi được. Thời gian sử dụng của một phương tiện được định nghĩa là thời điểm đến bến cuối trừ đi thời điểm bắt đầu (được cung cấp trước), (3) số lượng yêu cầu trong ngân hàng yêu cầu.
Ba thành phần được tính trọng số theo các hệ số $\alpha$, $\beta$, và $\gamma$ tương ứng. Thông thường, $\gamma$ sẽ được gán một giá trị lớn để phục vụ càng nhiều yêu cầu càng tốt. Một mô hình toán học sẽ được trình bày trong Phần 1 để xác định chính xác vấn đề.

Bài toán được lấy cảm hứng từ bài toán định tuyến phương tiện trong thực tế, liên quan đến việc vận chuyển nguyên liệu thô và hàng hóa giữa các cơ sở sản xuất của một nhà sản xuất thực phẩm lớn tại Đan Mạch. Vì lý do bảo mật, tác giả không thể trình bày bất kỳ dữ liệu nào về vấn đề thực tế đã thúc đẩy nghiên cứu này.

Đây là bài toán NP-khó (NP-hard), vì nó là trường hợp đặc biệt của bài toán tìm đường đi cho người giao hàng \textit{(traveling salesman problem)}. Mục tiêu của bài viết này là phát triển một phương pháp để tìm kiếm các giải pháp tốt, nhưng không nhất thiết là giải pháp tối ưu, cho bài toán được mô tả ở trên. Ưu tiên những phương pháp tương đối nhanh, "mạnh" và có thể xử lý các bài toán lớn. Vì vậy, có vẻ hợp lý khi sử dụng phương pháp heuristic.

Các phần tiếp theo sẽ khảo sát những nghiên cứu gần đây trên PDPTW. Mặc dù không có tài liệu tham khảo nào được đề cập dưới đây xem xét chính xác cùng một vấn đề như của tác giả, nhưng tất cả chúng đều gặp phải cùng một vấn đề cốt lõi.

Nanry \& Barnes (2000) là một trong số những người đầu tiên trình bày về metaheuristic cho PDPTW. Cách tiếp cận của họ dựa trên thuật toán tìm kiếm tabu động kết hợp một số vùng lân cận tiêu chuẩn (\textit{standard neighborhoods}). Để kiểm tra heuristic, Nanry và Barnes tạo ra các cấu hình PDPTW từ một tập hợp các bài toán định tuyến phương tiện tiêu chuẩn có ràng buộc thời gian (VRPTW) do Solomon đề xuất (1987). Heuristic được thử nghiệm trên các cấu hình có tối đa 50 yêu cầu. Li \& Lim (2001) sử dụng metaheuristic hỗn hợp (\textit{hybrid metaheuristic}) để giải quyết bài toán. Heuristic kết hợp thuật toán simulated annealing và tìm kiếm tabu. Phương pháp của họ được thử nghiệm trên 9 cấu hình lớn nhất của Nanry \& Barnes (2000), và họ xem xét 56 cấu hình mới dựa trên các bài toán VRPTW của Solomon (1987).
Lim, Lim, và Rodrigues (2002) áp dụng phương pháp tối ưu “\textit{squeaky wheel}” và tìm kiếm cục bộ cho PDPTW. Heuristic của họ được thử nghiệm trên một tệp các bài toán được đề xuất bởi Li \& Lim (2001). Lau \& Liang (2001) cũng áp dụng tìm kiếm tabu cho PDPTW, và họ mô tả một số cấu trúc heuristic cho bài toán này. Tác giả đặc biệt chú ý đến cách xây dựng các bài toán thử nghiệm từ các cấu hình VRPTW.

Gần đây, Bent và Van Hentenryck (2003a) đã đề xuất một heuristic cho bài toán PDPTW dựa trên phương pháp large neighborhood search. Heuristic đã được thử nghiệm trên các bài toán do Li \& Lim (2001) đề xuất. Heuristic của Bent và Van Hentenryck có lẽ là phương pháp metaheuristic hứa hẹn nhất cho PDPTW tính đến nay.

% Gendreau et al. (1998) xem xét một phiên bản động của bài toán. An ejection chain neighborhood is proposed, and steepest descent and tabu search heuristics based on the ejection chain neighborhood are tested. The tabu search is parallelized, and the sequential and parallelized versions are compare. (???)

Một số phương pháp tạo cột (column generation methods) cho PDPTW đã được đề xuất. Những phương pháp này bao gồm cả phương pháp chính xác và phương pháp sử dụng kinh nghiệm (heuristic). Dumas et al. (1991) là người đầu tiên sử dụng phương pháp tạo cột để giải bài toán PDPTW. Họ đề xuất một phương pháp phân nhánh và cận có thể xử lý các bài toán với tối đa 55 yêu cầu.

Xu et al. (2003) xét PDPTW với một số ràng buộc trong thực tế, bao gồm việc có nhiều khung ràng buộc thời gian, ràng buộc tương thích (compatibility constraints) và hạn chế về thời gian lái xe tối đa. Bài toán được giải quyết bằng cách sử dụng phương pháp tạo cột heuristic. Bài báo sẽ xét các bài toán với tối đa 500 yêu cầu.

Sigurd, Pisinger và Sig (2004) giải quyết bài toán PDPTW liên quan đến vận chuyển gia súc. Việc sinh ra một số ràng buộc bổ sung, chẳng hạn như sự ưu tiên giữa các yêu cầu, nghĩa là một số yêu cầu phải được đáp ứng trước những yêu cầu khác để tránh lây lan dịch bệnh. Cách giải quyết tối ưu của bài toán này là sử dụng phương pháp tạo cột. Bài toán lớn nhất có thể giải được chứa hơn 200 yêu cầu. Một cuộc khảo sát gần đây về vấn đề lấy và giao hàng được thực hiện bởi Desaulniers et al. (2002).

Nghiên cứu được trình bày trong bài báo này dựa trên luận văn thạc sĩ của Ropke (2002). Các bài báo của Pisinger \& Ropke (2005), và Ropke \& Pisinger (2006) đã chỉ ra lí do mà phương pháp heuristic được trình bày trong bài báo này có thể mở rộng để giải quyết nhiều vấn đề định tuyến phương tiện khác nhau — ví dụ VRPTW, \textit{bài toán định tuyến phương tiện đa điểm} , và \textit{bài toán định tuyến xe với Backhaul} ((Vehicle Routing Problem with Backhaul – VRPB)).

Phần còn lại của bài báo được sắp xếp như sau: Phần 1 đưa ra định nghĩa chính thức về bài toán PDPTW; phần 2 mô tả phương pháp giải cơ bản trong hoàn cảnh chung; phần 3 mô tả cách áp dụng phương pháp giải cho bài toán PDPTW, và trình bày phần mở rộng của của phương pháp. Phần 4 nói về kết quả của các tính toán thử nghiệm. Các tính toán này tập trung so sánh phương pháp heuristic với những phương pháp metaheuristic đã có, và đánh giá xem sự sàng lọc trong phần 3 có cải thiện phương pháp heuristic không.
