\chapter*{Giới thiệu}
Trong biến thể được xem xét của vấn đề giao và nhận hàng với thời gian ràng buộc (PDPTW), tác giả được cung cấp một số lượng yêu cầu và phương tiện. Một yêu cầu bao gồm nhận hàng tại một địa điểm và giao những hàng hóa này đến một địa điểm khác. Hai khung giờ ràng buộc được chỉ định cho mỗi yêu cầu: khung giờ ràng buộc lấy hàng chỉ định thời điểm có thể nhận hàng và khung giờ ràng buộc giao hàng cho biết khi nào hàng hóa có thể được chuyển xuống. Hơn nữa, thời gian phục vụ được liên kết với mỗi lần nhận và giao hàng. Thời gian dịch vụ cho biết sẽ mất bao lâu để thực hiện việc nhận hàng hoặc giao hàng. Một phương tiện được phép đến một địa điểm trước khi khung giờ ràng buộc của địa điểm đó bắt đầu, nhưng sau đó phương tiện phải đợi cho đến khi khung giờ ràng buộc bắt đầu mới bắt đầu hoạt động. Một phương tiện có thể không bao giờ đến một địa điểm sau khi kết thúc thời gian ràng buộc của địa điểm đó.

Mỗi yêu cầu được chỉ định một tập hợp các phương tiện khả thi. Ví dụ, điều này có thể được sử dụng để mô hình hóa các tình huống trong đó một số phương tiện không thể đi vào một vị trí nhất định do kích thước của phương tiện.

Mỗi phương tiện có một sức chứa giới hạn, và nó bắt đầu và kết thúc nhiệm vụ của mình tại các địa điểm nhất định được gọi là bến đầu và bến cuối. Vị trí xuất phát và kết thúc không nhất thiết phải giống nhau và hai phương tiện có thể có bến xuất phát và kết thúc khác nhau.
Hơn nữa, mỗi chiếc xe được chỉ định một thời gian bắt đầu và kết thúc. Thời gian bắt đầu cho biết thời điểm phương tiện phải rời khỏi vị trí bắt đầu và thời gian kết thúc biểu thị thời gian đến vị trí cuối cùng được phép muộn nhất. Lưu ý rằng phương tiện rời khỏi kho vào thời gian bắt đầu được chỉ định mặc dù điều này có thể gây ra thời gian chờ tại địa điểm đầu tiên được truy cập.

Nhiệm vụ của tác giả là xây dựng các tuyến đường hợp lệ cho các phương tiện. Một tuyến đường hợp lệ nếu các thời gian ràng buộc và các giới hạn về sức chứa được tuân thủ dọc theo tuyến đường, mỗi lần lấy hàng được phục vụ trước khi giao hàng tương ứng, việc nhận hàng và giao hàng tương ứng được phục vụ trên cùng một tuyến đường và phương tiện chỉ phục vụ các yêu cầu mà nó được phép phục vụ. Các tuyến đường nên được xây dựng sao cho chúng giảm thiểu cost function được mô tả dưới đây.

Vì số lượng phương tiện có hạn, tác giả có thể gặp phải tình huống không thể chỉ định một số yêu cầu cho phương tiện. Những yêu cầu này được đặt trong một \textit{request bank} ảo. Trong tình huống thực tế, người điều hành là người quyết định phải làm gì với những yêu cầu như vậy. Ví dụ, nhà điều hành có thể quyết định thuê thêm xe để phục vụ các yêu cầu còn lại.

Mục tiêu của bài toán là cực tiểu hóa tổng trọng số bao gồm ba thành phần sau: (1) tổng quãng đường các xe đã đi, (2) tổng thời gian mỗi xe đi được. Thời gian sử dụng của một phương tiện được định nghĩa là thời gian đến bến cuối trừ đi thời gian bắt đầu (được cung cấp trước), (3) số lượng yêu cầu trong \textit{request bank}.
Ba thuật ngữ được tính trọng số theo các hệ số $\alpha$, $\beta$, và $\gamma$ tương ứng. Thông thường, một giá trị cao được chỉ định để phục vụ càng nhiều yêu cầu càng tốt. Một mô hình toán học được trình bày trong §1 để xác định chính xác vấn đề

Bài toán được lấy cảm hứng từ một bài toán định tuyến phương tiện thực tế liên quan đến việc vận chuyển nguyên liệu thô và hàng hóa giữa các cơ sở sản xuất của một nhà sản xuất thực phẩm lớn của Đan Mạch. Vì lý do bảo mật, tác giả không thể trình bày bất kỳ dữ liệu nào về vấn đề thực tế đã thúc đẩy nghiên cứu này.

Vấn đề là NP-hard, vì nó chứa vấn đề người bán hàng du lịch như một trường hợp đặc biệt. Mục tiêu của bài viết này là phát triển một phương pháp để tìm kiếm các giải pháp tốt, nhưng không nhất thiết là tối ưu, cho vấn đề được mô tả ở trên. Phương pháp được phát triển tốt nhất nên tương đối nhanh, mạnh mẽ và có thể xử lý các vấn đề lớn. Vì vậy, có vẻ công bằng khi chuyển sang các phương pháp heuristic.

Các đoạn tiếp theo khảo sát công việc gần đây trên PDPTW. Mặc dù không có tài liệu tham khảo nào được đề cập dưới đây xem xét chính xác cùng một vấn đề như của tác giả, nhưng tất cả chúng đều gặp phải cùng một vấn đề cốt lõi.

Nanry và Barnes (2000) là một trong số những người đầu tiên trình bày một metaheuristic cho PDPTW. Cách tiếp cận của họ dựa trên thuật toán tìm kiếm tabu phản ứng kết hợp một số vùng lân cận tiêu chuẩn. Để kiểm tra heuristic, Nanry và Barnes tạo ra các cá thể PDPTW từ một tập hợp các bài toán định tuyến phương tiện tiêu chuẩn với các bài toán về ràng buộc thời gian (VRPTW) do Solomon đề xuất (1987). Heuristic được thử nghiệm trên các phiên bản có tối đa 50 yêu cầu. Li và Lim (2001) sử dụng siêu dữ liệu lai ghép để giải quyết vấn đề. Heuristic kết hợp ủ mô phỏng và tìm kiếm tabu. Phương pháp của họ được thử nghiệm trên 9 trường hợp lớn nhất từ Nanry và Barnes (2000), và họ xem xét 56 trường hợp mới dựa trên các vấn đề VRPTW của Solomon (1987).
Lim, Lim, và Rodrigues (2002) áp dụng tối ưu hóa “\textit{squeaky wheel}” và tìm kiếm cục bộ cho PDPTW. Heuristic của họ được thử nghiệm trên tập hợp các vấn đề được đề xuất bởi Li và Lim (2001). Lau và Liang (2001) cũng áp dụng tìm kiếm tabu cho PDPTW, và họ mô tả một số kinh nghiệm xây dựng cho vấn đề này. tác giả đặc biệt chú ý đến cách xây dựng các vấn đề thử nghiệm từ các phiên bản VRPTW.

Gần đây, Bent và Van Hentenryck (2003a) đã đề xuất một heuristic cho PDPTW dựa trên tìm kiếm vùng lân cận lớn. Heuristic đã được thử nghiệm trên các vấn đề được đề xuất bởi Li và Lim (2001). Heuristic của Bent và Van Hentenryck có lẽ là metaheuristic hứa hẹn nhất cho PDPTW được đề xuất cho đến nay.

Gendreau et al. (1998) xem xét một phiên bản động của vấn đề. Một vùng lân cận chuỗi phóng được đề xuất, và các phương pháp phỏng đoán tìm kiếm tabu và suy giảm độ dốc heuristic dựa trên vùng lân cận chuỗi phóng được thử nghiệm. Tìm kiếm tabu được song song hóa và các phiên bản tuần tự và song song được so sánh

Một số phương pháp tạo cột cho PDPTW đã được đề xuất. Những phương pháp này bao gồm cả phương pháp chính xác và heuristic. Dumas et al. (1991) là người đầu tiên sử dụng việc tạo cột để giải PDPTW. Họ đề xuất một phương pháp phân nhánh và giới hạn có thể xử lý các sự cố với tối đa 55 yêu cầu.

Xu et al. (2003) xem xét PDPTW với một số ràng buộc thực tế bổ sung, bao gồm nhiều thời gian ràng buộc, hạn chế tương thích và hạn chế thời gian lái xe tối đa. Vấn đề được giải quyết bằng cách sử dụng heuristic tạo cột. Bài báo xem xét các trường hợp sự cố với tối đa 500 yêu cầu.

Sigurd, Pisinger và Sig (2004) giải quyết vấn đề PDPTW liên quan đến vận chuyển gia súc. Điều này đưa ra một số ràng buộc bổ sung, chẳng hạn như mối quan hệ ưu tiên giữa các yêu cầu, nghĩa là một số yêu cầu phải được đáp ứng trước những yêu cầu khác để tránh lây lan dịch bệnh. Vấn đề được giải quyết tối ưu bằng cách tạo cột. Các vấn đề lớn nhất được giải quyết chứa hơn 200 yêu cầu. Một cuộc khảo sát gần đây về tài liệu về vấn đề nhận và giao hàng đã được thực hiện bởi Desaulniers et al. (2002).

Công việc được trình bày trong bài báo này dựa trên luận văn thạc sĩ của Ropke (2002). Trong các bài báo của Pisinger và Ropke (2005) và Ropke và Pisinger (2006), nó đã chỉ ra cách mà heuristic được trình bày trong bài báo này có thể được mở rộng để giải quyết nhiều vấn đề định tuyến phương tiện khác nhau—ví dụ VRPTW, bài toán định tuyến phương tiện đa điểm , và vấn đề định tuyến phương tiện với các chuyến lùi.
